\section{Technische Produktumgebung}
    \subsection{Hardware}
        \begin{itemize}
            \item (mobiles) Gerät mit Internetzugang.
            \item 100MB freier Speicher auf dem Gerät
        \end{itemize}
    \subsection{Software}
        \paragraph{Frontend}
            Das \Gls{Frontend} ist eine \Gls{PWA} und ist über einen beliebigen \Gls{Browser} zu erreichen.
            Es basiert auf dem  \Gls{Framework} \Gls{Vue.js} und ist in \Gls{TypeScript} und \Gls{HTML5} geschrieben.
            
        \paragraph{Backend}
            Das \Gls{Server}-\Gls{Backend} läuft auf einer virtuellen Maschine (\Gls{VM}) auf einem \Gls{Server}.
            Auf der \Gls{VM} läuft Linux Ubuntu 20.04 mit der Software \Gls{Spring-Boot} als \Gls{Backend}-\Gls{Framework}.
            Das \Gls{Backend} ist in \Gls{Java} 17 geschrieben.
            Die Daten werden in einer \Gls{MySQL}-\Gls{Datenbank} gespeichert.
    \subsection{Schnittstellen}
        Die Benutzer\gls{Schnittstelle} wird über eine \Gls{GUI} zur Verfügung gestellt.
        Das \Gls{Frontend} kommuniziert über eine \Gls{Rest}-\Gls{API} mit dem \Gls{Backend}.
        
            
