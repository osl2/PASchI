\section{Produktübersicht}

Um sich einen Überblick über das Produkt zu verschaffen, werden im folgenden Use-Case-Diagramm (\autoref{fig:usecase}) die grundlegenden Funktionalitäten beschrieben. Die Abschnitte \frqq Sitzung eröffnen\flqq{}, \frqq Schüler verwalten\flqq{} und \frqq Kurse verwalten\flqq{} werden jeweils in einem Aktivitätsdiagramm genauer beschrieben.

\paragraph{Allgemeiner Aufbau (\autoref{fig:usecase})} $~$ 

Das Use-Case-Diagramm \frqq Allgemeiner Aufbau\flqq{} bietet eine Übersicht über die grundlegenden Produktfunktionen und Anwendungsfälle. Wird die Anwendung zum ersten Mal geöffnet, muss der Nutzer sich mit seinem Namen, seiner E-Mail-Adresse sowie einem Passwort registrieren. Nach erfolgreichem \Gls{Login} landet der Nutzer auf der Startseite. Von dort aus stehen dem Nutzer vier verschiedene Menüpunkte zur Auswahl: In der \textit{Schülerbibliothek} kann er seine angelegten Schüler in einer Liste ansehen und verwalten (erstellen, löschen) und die Statistiken eines Schülers anzeigen lassen. Analog kann der Nutzer seine Kurse in der \textit{Kursbibliothek} ansehen und verwalten sowie Schüler zu Kursen hinzufügen und aus ihnen entfernen. Wurde ein Kurs mit Schülern angelegt, kann der Nutzer eine neue \textit{Sitzung} erstellen und Interaktionen von Schülern aufzeichnen. Nach Beenden der Sitzung kann die erstellte Interaktionskarte in den \textit{Aufzeichnungen} zusammen mit weiteren Statistiken angesehen sowie exportiert werden.

\paragraph{Interaktionsaufzeichnung (\autoref{fig:activity1})} $~$ 

Das Aktivitätsdiagramm \frqq Interaktionsaufzeichnung\flqq{} beschreibt den Ablauf einer Sitzung. Wird von der Startseite aus eine neue Sitzung erstellt, muss der Nutzer zunächst einen Kurs auswählen. Danach können Interaktionen zwischen Schülern sowie zwischen Schülern und Lehrern aufgezeichnet werden. Interaktionen müssen im Anschluss einer Kategorie (z.B. Störung, Meldung, Antwort, etc.) zugeordnet werden. Wird die Sitzung beendet, hat der Nutzer die Möglichkeit, die soeben erstellte Interaktionskarte zu exportieren. Erstellte Interaktionskarten können in den Aufzeichnungen angesehen werden.

\paragraph{Schüler verwalten (\autoref{fig:activity2})} $~$ 

Das Aktivitätsdiagramm \frqq Schüler verwalten\flqq{} beschreibt die Abläufe, um Schüler zu verwalten. Öffnet der Nutzer die Schülerbibliothek, wird ihm zunächst eine Liste der bereits angelegten Schüler angezeigt. Von dort aus kann er nun neue Schüler erstellen oder einen bestehenden Schüler auswählen und Statistiken über den Schüler ansehen. Hier hat der Nutzer auch die Möglichkeit, einen ausgewählten Schüler zu löschen. Wird ein neuer Schüler erstellt, müssen seine Daten eingegeben werden (Name, etc.). Optional kann der neu erstellte Schüler direkt zu Kursen hinzugefügt werden.

\paragraph{Kurse verwalten (\autoref{fig:activity3})} $~$ 

Das Aktivitätsdiagramm \frqq Kurse verwalten\flqq{} beschreibt die Abläufe, um Kurse zu verwalten. In der Kursbibliothek können, wie auch in der Schülerbibliothek, neue Kurse angelegt, sowie bestehende Kurse mitsamt deren Statistiken angesehen und verwaltet werden. Wird ein bestehender Kurs ausgewählt, kann er entweder gelöscht werden oder man kann eine Liste der Kursteilnehmer anzeigen lassen. Von dort aus können Schüler zum Kurs hinzugefügt oder aus ihm entfernt werden. Wird ein neuer Kurs erstellt, müssen die Daten des Kurses eingegeben werden. Optional können Schüler direkt dem neu erstellten Kurs hinzugefügt werden.

\paragraph{Nutzerstatistiken} $~$ 

Der Nutzer hat die Möglichkeit, sich verschiedene Statistiken zu den angelegten Schülern und Kursen sowie vergangenen Sitzungen anzeigen zu lassen. Zu den Schülerstatistiken gehören beispielsweise die durchschnittliche Qualität der Interaktionen eines Schülers und die prozentuale Aufschlüsselung der Interaktionen nach der Kategorie. Die Kursstatistiken beinhalten eine Auflistung der Schüler, die sich am häufigsten bzw. seltensten am Unterricht beteiligen. Außerdem gibt es eine Liste mit den Schülern, die den Unterricht am häufigsten stören. In den Sitzungsstatistiken werden neben der Interaktionskarte der Sitzung auch Statistiken wie die Anzahl der Interaktionen pro Kategorie dargestellt. Zudem kann es Statistiken zur Beteiligungsquote geben. Hier können dann ebenfalls Schüler hervorgehoben werden, die sich besonders viel bzw. wenig beteiligen oder den Unterricht stören.


\begin{figure}[ht]
    \centering
    \includesvg[inkscapelatex=false,width=1\columnwidth]{graphics/use_case_diagramm.svg}
    \caption{Allgemeiner Aufbau}
    \label{fig:usecase}
\end{figure}

\begin{figure}[ht]
    \centering
    \includesvg[inkscapelatex=false,width=0.6\columnwidth]{graphics/Aktivitätsdiagramm1.svg}
    \caption{Interaktionsaufzeichnung}
    \label{fig:activity1}
\end{figure}

\begin{figure}[ht]
    \centering
    \includesvg[inkscapelatex=false,width=1\columnwidth]{graphics/Aktivitätsdiagramm2.svg}
    \caption{Schüler verwalten}
    \label{fig:activity2}
\end{figure}

\begin{figure}[ht]
    \centering
    \includesvg[inkscapelatex=false,width=1\columnwidth]{graphics/Aktivitätsdiagramm3.svg}
    \caption{Kurse verwalten}
    \label{fig:activity3}
\end{figure}