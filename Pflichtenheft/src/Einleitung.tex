\section{Einleitung}
    Im Klassenzimmer den Überblick zu behalten, ist für Lehrkräfte eine große Herausforderung. Vor allem,
    wenn es dann um Einschätzungen von Leistungspotential und Mitarbeit einzelner Schüler:innen (z.B. für
    mündliche Noten) sowie von Rahmenbedingungen wie Gruppen- und Klassendynamik (z.B. Störfaktoren)
    geht. Hier müssen sich die meisten dann auf ihre subjektive Wahrnehmung und ihr Erinnerungsvermögen
    verlassen. Dabei spielen gerade solche Aspekte eine wichtige Rolle dabei, eine Lernatmosphäre zu schaffen,
    von der alle Lernenden optimal profitieren. Hierzu braucht es neue Möglichkeiten zur Erfassung des Unterrichtsgeschehens, die eine objektivere Einschätzung der Unterrichts- und Lernsituation auch im Nachhinein
    ermöglichen. Solche Tools müssen aber gleichzeitig unkompliziert und einfach in der Handhabung sein.
    In diesem Projekt soll eine Interaktions-App entwickelt werden, die das Classroom-Management für Lehrkräfte im oben umrissenen Nutzfall vereinfachen soll.